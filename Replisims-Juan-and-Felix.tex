\documentclass[10,a4paperpaper,]{article}

  \title{Replication Report Brookhart et al. (2006)}
  \author{Juan Claramunt\textsuperscript{1}, \and Felix
Clouth\textsuperscript{2}}
  \date{%
		\textsuperscript{1} Leiden University\\%
		\textsuperscript{2} Tilburg University~\\[2ex]
		\today
   }
  


\newcommand{\iblue}{008080}
\newcommand{\igray}{d4dbde}

\newlength{\cslhangindent}
\setlength{\cslhangindent}{1.5em}
\newenvironment{CSLReferences}%
  {}%
  {\par}

% Author: Karol KozioL
% License: GPL-3
% Modified by: Sarah Wagner & Anna Lohmann

% % % packages -----------------------------------------------------------------------------------
\usepackage{amsmath}
\usepackage{array}
\usepackage{booktabs}
\usepackage{calc}
\usepackage{eso-pic}
\usepackage{fancyhdr}
\usepackage{fontspec}
\usepackage[left = 2.5cm, right = 2.5cm, top = 1.2cm, bottom = 1.2cm, includeheadfoot]{geometry}
\usepackage{graphicx}
\usepackage[utf8]{inputenc}
\usepackage{lastpage}
\usepackage{multirow}
\usepackage{tabularx} 
\usepackage{tikz}
\usepackage{titlesec}
\usepackage{xcolor, colortbl}
\usepackage{url} 
\usepackage[hidelinks]{hyperref} 
\usepackage{pmboxdraw}
\usepackage{placeins}
\usepackage{enumitem}
\usepackage{longtable}
\usepackage{lscape}
%\usepackage{verbatim}
%\makeatletter
%\def\verbatim@font{\scriptsize\ttfamily}
%\makeatother
%\RequirePackage[normalem]{ulem} %DIF PREAMBLE
%\RequirePackage{color}
%\providecommand{\tightlist}{%
%	\setlength{\itemsep}{0pt}\setlength{\parskip{0pt}}
\providecommand{\tightlist}{%
  \setlength{\itemsep}{0pt}\setlength{\parskip}{0pt}}
% % % settings -----------------------------------------------------------------------------------

% % custom colors
\definecolor{iblue}{HTML}{\iblue}
\definecolor{igray}{HTML}{\igray}

% definition of pagename
\newcommand\pagename{Page}

% % fonts 
\defaultfontfeatures{Mapping = tex-text}
\setmainfont[BoldFont = Lato-Bold.ttf, ItalicFont = Lato-Italic.ttf, BoldItalicFont = Lato-BoldItalic.ttf]{Lato-Regular.ttf}
\newfontfamily\headingfont[ItalicFont = Lato-BlackItalic.ttf]{Lato-Black.ttf}
%\setmonofont{Ubuntu Mono}
\setmonofont[Scale=0.90,
BoldFont=UbuntuMono-Bold.ttf,
%ItalicFont=UbuntuMono-Italic.ttf,
BoldItalicFont=UbuntuMono-BoldItalic.ttf
]{UbuntuMono-Regular.ttf}

\makeatletter
\def\verbatim@font{\linespread{1}\normalfont\ttfamily}
\makeatother

% % sections
\titleformat{\section}{\color{iblue}\headingfont\Large\bfseries}{\thesection}{1em}{}[\titlerule]
\titleformat{\subsection}{\color{iblue}\headingfont\large\bfseries}{\thesubsection}{1em}{}
\titleformat{\subsubsection}{\color{iblue}\headingfont\bfseries}{\thesubsubsection}{1em}{}

% % misc
\setlength{\parindent}{0em} 
\linespread{1.5}
\raggedright
\newcolumntype{C}{>{\centering\arraybackslash}X}


\makeatletter

% pagestyle titlepage
\fancypagestyle{customtitle}{
	\lhead{}
	\chead{}
	\rhead{}
	\makeatother
	\lfoot{}
	\cfoot{}
	\rfoot{}
}




% % % header and footer ---------------------------------------------------------------------------
\pagestyle{fancy}
\lhead{}
\chead{}
\rhead{}
\makeatother
\newlength{\myheight}
\lfoot{}
\cfoot{}
\rfoot{\pagename~\thepage \hspace{1pt} / \pageref{LastPage}}
\renewcommand\headrulewidth{0pt}
\renewcommand\footrulewidth{0pt}





\begin{document}


\renewcommand{\contentsname}{Table of Contents}

\renewcommand{\pagename}{Page}


\urlstyle{same}

\maketitle

Replication Report, Brookhart et al.~(2006)

\subsection*{Abstract}

In this document, we describe a replication study of ``Variable
Selection for Propensity Score Models'' by Brookhart et al.~(Brookhart
et al. 2006). This report includes an analysis of the data-generating
mechanism, simulation study set-up and methods described in the original
paper, the feasibility of its implementation based on the information
available in the original document and the results we were able to
replicate. Furthermore, we analyze whether the presentation of the
methods and results on the original paper can be improved. We were able
to replicate the results of the first experiment of the original paper
by making a few small assumptions. However, we were not able to
replicate the results of experiment two. This was probably caused by a
lack of explanation of certain concepts used on the article, the lack of
experience of the authors of this report and the extra difficulty that
requires replicating a figure instead of a table.

\vskip 2em

\noindent\makebox[\textwidth]{\large Correspondence concerning this replication report should be addressed to:}

\par

\noindent\makebox[\textwidth]{\large f.j.clouth@tilburguniversity.edu}

\par

\clearpage

\section{Introduction}

This replication report documents the replication attempt of the
simulation study Brookhart et al. (2006). Following the definition of
Rougier et al. (2017) we understand the replication of a published study
as writing and running new code based on the description provided in the
original publication with the aim of obtaining the same results. This
replication study was independently done by another pair of researchers.
In the discussion section we will include an overview of the common
findings an differences.

\section{Method}

\subsection{Information basis}

This replication was conducted mainly on information provided in
Brookhart et al. (2006). Software code or additional information (such
as supplementary materials) were not available. The addition of the open
source code (according to the original paper, it exists but it is not
published) will make the replication of the study much easier.
Additionally, we used information provided in the splines2 and ROCR
R-package documentation. However, this information was not referred to
in the main article.

\subsection{Data Generating Mechanism}

Information provided in the above mentioned sources indicated that the
following simulation factors were systematically varied in generating
the artificial data.

\subsubsection{Experiment 11}

\begin{longtable}[]{@{}
  >{\raggedright\arraybackslash}p{(\columnwidth - 4\tabcolsep) * \real{0.19}}
  >{\raggedright\arraybackslash}p{(\columnwidth - 4\tabcolsep) * \real{0.06}}
  >{\raggedright\arraybackslash}p{(\columnwidth - 4\tabcolsep) * \real{0.74}}@{}}
\toprule
Simulation factor & N o . l e v e l s & Levels \\
\midrule
\endhead
\emph{Varied} & & \\
Sample size & 2 & 500, 2500 \\
\emph{Fixed} & & \\
\(\alpha_{0}\) & & 0.5 \\
\(\alpha_{1}\) & & 4 \\
\(\alpha_{2}\) & & 1 \\
\(\alpha_{3}\) & & 0 \\
\(\alpha_{4}\) & & 0.5 \\
\(\beta_{0}\) & & 0 \\
\(\beta_{1}\) & & 0.5 \\
\(\beta_{2}\) & & 0 \\
\(\beta_{3}\) & & 0.75 \\
Dichotomous exposure A & & sampled from rbern(N,
prob=pnorm(\(q=\beta_{0} + \beta_{1}*X_{1}+\beta_{2}*X_{2}+\beta_{3}*X_{3}\))) \\
Outcome Y & & sampled from rpois(N,
lamda=\(exp(\alpha_{0}+\alpha_{1}*(((1+exp(-3*X_{1}))^{-1})-0.5)+\alpha_{2}*X_{2}+\alpha_{3}*X_{3}+\alpha_{4}*A))\) \\
Three covariates & & independently sampled from rnorm(N, 0, 1) \\
\bottomrule
\end{longtable}

\subsubsection{Experiment 1 (sensitivity analysis)}

\begin{longtable}[]{@{}
  >{\raggedright\arraybackslash}p{(\columnwidth - 4\tabcolsep) * \real{0.19}}
  >{\raggedright\arraybackslash}p{(\columnwidth - 4\tabcolsep) * \real{0.06}}
  >{\raggedright\arraybackslash}p{(\columnwidth - 4\tabcolsep) * \real{0.74}}@{}}
\toprule
Simulation factor & N o . l e v e l s & Levels \\
\midrule
\endhead
\emph{Varied} & & \\
Standard deviation of covariates & 2 & 0.5, 1.5 \\
\(\alpha_{4}\) & 3 & 0.25, 0.5, 1 \\
\(\beta_{0}\) & 2 & -1, 0 \\
\emph{Fixed} & & \\
Sample size & & 500 \\
\(\alpha_{0}\) & & 0.5 \\
\(\alpha_{1}\) & & 4 \\
\(\alpha_{2}\) & & 1 \\
\(\alpha_{3}\) & & 0 \\
\(\beta_{1}\) & & 0.5 \\
\(\beta_{2}\) & & 0 \\
\(\beta_{3}\) & & 0.75 \\
Dichotomous exposure A & & sampled from rbern(N,
prob=pnorm(\(q=\beta_{0} + \beta_{1}*X_{1}+\beta_{2}*X_{2}+\beta_{3}*X_{3}\))) \\
Outcome Y & & sampled from rpois(N,
lamda=\(exp(\alpha_{0}+*\alpha_{1}*(((1+exp(-3*X_{1}))^{-1})-0.5)+\alpha_{2}*X_{2}+\alpha_{3}*X_{3}+\alpha_{4}*A))\) \\
\bottomrule
\end{longtable}

\subsubsection{Experiment 2}

\begin{longtable}[]{@{}
  >{\raggedright\arraybackslash}p{(\columnwidth - 4\tabcolsep) * \real{0.19}}
  >{\raggedright\arraybackslash}p{(\columnwidth - 4\tabcolsep) * \real{0.06}}
  >{\raggedright\arraybackslash}p{(\columnwidth - 4\tabcolsep) * \real{0.74}}@{}}
\toprule
Simulation factor & N o . l e v e l s & Levels \\
\midrule
\endhead
\emph{Varied} & & \\
Sample size & 2 & 500, 2500 \\
\(\alpha_{1}\) & 2 1 & {[}0, .01, \ldots, .2{]} \\
\(\beta_{1}\) & 2 6 & {[}0, .05, \ldots, 1.25{]} \\
\emph{Fixed} & & \\
\(\alpha_{0}\) & & 0.5 \\
\(\alpha_{2}\) & & 1 \\
\(\alpha_{3}\) & & 0 \\
\(\alpha_{4}\) & & 0.5 \\
\(\beta_{0}\) & & 0 \\
\(\beta_{2}\) & & 0 \\
\(\beta_{3}\) & & 0.75 \\
Three covariates & & independently sampled from rnorm(N, 0, 1) \\
Outcome Y & & sampled from
rpois(exp(\(\alpha_{0}+\alpha_{1}*X_{1 }+\alpha_{2}*X_{2}+\alpha_{3}*X_{3}+\alpha_{4}*A\))) \\
Dichotomous exposure A & & sampled from rbern(N,
prob=pnorm(\(q=\beta_{0} + \beta_{1}*X_{1}+\beta_{2}*X_{2}+\beta_{3}*X_{3}\))) \\
\bottomrule
\end{longtable}

The design was full-factorial in the main analyses of Experiment 1 and
Experiment 2. For the sensitivity analyses of Experiment 1, all
parameters were held at their default values while a single parameter
was altered.

\subsection{Research Comparison}

The study compares different specifications of a propensity scores (PS)
model using a Monte-Carlo simulation. The model examines the effects of
the inclusion of three types of covariates: X1 (a true confounder), X2
(a predictor of the outcome) or X3 (a variable related to the exposure
but not the outcome).

The exposure effects are estimated using two methods: (1) by adjusting
for the PS in a multivariate outcome model in which the effect of the
estimated PS was modeled through a regression spline. In this case, the
propensity score model fit is given by the following formula:

\[
E[Y|\hat{PS},A] = exp \{ \lambda + \sum_k \psi_k B_k(\hat{PS}) + \gamma A \}
\]

where \(\lambda\) is the baseline rate, \(B_k\) are B-spline basis
functions and \(\gamma\) is the treatment effect. (2) Using
subclassification in which strata are defined by quintiles of the
estimated propensity scores and taking the average treatment effect
across strata.

\subsection{Performance measures}

To measure the performance of the different specifications of the PS
models of the original simulation study, the authors measured the bias,
the variance and the mean square error of the estimates of the parameter
\(\alpha_{4}\) and the average C-stat.

\subsection{Technical implementation}

While the original simulation study was carried out in R version 1.9.1
running on a Windows XP platform, our replication was implemented using
the R programming environment version 4.0.3 on a Windows 10 environment
(details regarding software versions can be obtained from the section
Reproducibility Information). The corresponding R code can be obtained
from \url{https://github.com/replisims/your_repo}.

The following table provides an overview of replicator degrees of
freedom, i.e.~decisions that had to be made by the replicators because
of insufficient or contradicting information. Issues were resolved by
discussion among the replicators. Decisions were based on what the
replicators perceived to be the most likely implementation with
likeliness estimated by common practice and/or guideline
recommendations. Wherever feasible multiple interpretations where
implemented.

\begin{longtable}[]{@{}
  >{\raggedright\arraybackslash}p{(\columnwidth - 4\tabcolsep) * \real{0.36}}
  >{\raggedright\arraybackslash}p{(\columnwidth - 4\tabcolsep) * \real{0.36}}
  >{\raggedright\arraybackslash}p{(\columnwidth - 4\tabcolsep) * \real{0.26}}@{}}
\toprule
Issue & Replicator decision & Justification \\
\midrule
\endhead
It was not clear whether to estimate the PS or to calculate them from
the population model & Estimate the PS & It should result in a better
fit for each simulated dataset. \\
The implementation of the cubic spline estimation was not clearly
explained. Moreover, the reference included does not provide enough
insight & We used the function bs() in splines

package, with specifying three

knots splines::bs(PS, knots =

quantile(PS, probs = c(0.25, 0.5,

0.75))) & Only available function in R. We used the default options. \\
Subclassification approach not fully defined. Not clear how to obatin
the estimates of the exposure rate within strata. & We used the mean as
estimator & It is the most reasonable choice. \\
No description of c-statistic & Hmisc::somers2(PS,

data{[},``exposure''{]}){[}``C''{]} & Most likely meaning of the
c-statistic. Most common R package to compute the area under the ROC
curve. \\
Some values required for experiment 2 were missing & Used the same
values we used in experiment 1 & If the values are omitted, the most
likely reason is that the values are the same for both experiments. \\
Definition and scales of the variables measured in the axes of the
figures of experiment 2 were not clearly described. & No solution & \\
Estimators of experiment 2 were not clearly explained. & Use the
estimators of experiment 1 that fitted better the brief explanation & It
is the most reasonable choice. \\
According to the Methods section, the PS was adjusted using the spline
function in experiment 2. However, in the results section we can find
the quintile approach as well. & No solution & \\
\bottomrule
\end{longtable}

\subsection{C-statistic issue}

In a first version of the original paper we were trying to replicate, it
was not specified what the \emph{c} statistic mean. Therefore, it was
not clear what measure we were replicating nor how we were supposed to
do so. In latter versions, it is stated that the \emph{c}-statistics
they use in this research is the area under the receiver operating
characteristic (ROC) curve.

\subsection{Experiment 2 issue}

Whilw trying to replicate the results of the second experiment we found
multiple issues. First, they use multiple transformation of the results
that are not defined in the paper. Second, it is far more difficult to
compare the results obtained in a graph than in a table. While we could
easily check that the results we obtained for experiment 1 were exactly
the same as the results obtained in the original paper, it was
incredibly difficult to do the same with experiment 2. Furthermore,
there is no open code available to replicate these figures. These
difficulties did not allow us to replicate the figures of experiment 2.

\section{Results}

\subsection{Experiment 1}

In this section we show the tables we obtain during the replication of
experiment 1.

\begin{table}
\caption{Table 1. Simulation experiment 1, with results based on an analysis in which the propensity score is entered into an outcome model as a parametric spline term.}
\begin{tabular}{l r r r r r r r r}
\hline
\multicolumn{8}{c}{Variable(s) in propensity score model}\\
\hline
  & $X_{1}$ & $X_{2}$ & $X_{3}$ & $X_{1}$, $X_{2}$ & $X_{1}$, $X_{3}$ & $X_{2}$, $X_{3}$ & $X_{1}$, $X_{2}$, $X_{3}$ & None\\
\hline
n = 500\\

Bias & 0.008 & 0.601 & 0.744 & 0.006 & 0.007 & 0.744 & 0.006 & 0.599\\

Variance & 0.030 & 0.025 & 0.047 & 0.020 & 0.040 & 0.038 & 0.030 & 0.040\\

MSE & 0.030 & 0.386 & 0.600 & 0.020 & 0.040 & 0.591 & 0.030 & 0.398\\

avg c-stat & 0.67 & 0.52 & 0.76 & 0.67 & 0.82 & 0.76 & 0.82 & \\

n = 2500\\

Bias & 0.008 & 0.601 & 0.744 & 0.006 & 0.007 & 0.744 & 0.006 & 0.592\\

Variance & 0.030 & 0.025 & 0.047 & 0.020 & 0.040 & 0.038 & 0.030 & 0.009\\

MSE & 0.030 & 0.386 & 0.600 & 0.020 & 0.040 & 0.591 & 0.030 & 0.359\\

avg c-stat & 0.67 & 0.51 & 0.76 & 0.67 & 0.81 & 0.76 & 0.81 & \\
\hline
\end{tabular}
\end{table}

\begin{table}
\caption{Table 2. Simulation experiment 1, with results based on an analysis using
subclassification in which strata are defined by quintiles of the estimated propensity
score.}
\begin{tabular}{l r r r r r r r r}
\hline
\multicolumn{8}{c}{Variable(s) in propensity score model}\\
\hline
  & $X_{1}$ & $X_{2}$ & $X_{3}$ & $X_{1}$, $X_{2}$ & $X_{1}$, $X_{3}$ & $X_{2}$, $X_{3}$ & $X_{1}$, $X_{2}$, $X_{3}$ & None\\
\hline
n = 500\\

Bias & 0.027 & 0.607 & 0.802 & 0.030 & 0.037 & 0.801 & 0.033 & 0.599\\

Variance & 0.022 & 0.018 & 0.050 & 0.016 & 0.048 & 0.039 & 0.038 & 0.040\\

MSE & 0.023 & 0.386 & 0.693 & 0.016 & 0.050 & 0.680 & 0.039 & 0.398\\

n = 2500\\

Bias & 0.029 & 0.597 & 0.763 & 0.029 & 0.063 & 0.762 & 0.061 & 0.592\\

Variance & 0.004 & 0.003 & 0.011 & 0.003 & 0.012 & 0.009 & 0.010 & 0.009\\

MSE & 0.005 & 0.360 & 0.594 & 0.004 & 0.016 & 0.589 & 0.014 & 0.359\\
\hline
\end{tabular}
\end{table}

\begin{table}
\caption{Table 3. Sensitivity analysis of simulation study 1. We consider nine different perturbations of the simulation parameters. Results are from 1,000 simulations of
data (n = 500), using a parametric spline to adjust for the estimated propensity score.}
\begin{tabular}{l r r r r r r r r}
\hline
\multicolumn{8}{c}{Variable(s) in propensity score model}\\
\hline
  & $X_{1}$ & $X_{2}$ & $X_{3}$ & $X_{1}$, $X_{2}$ & $X_{1}$, $X_{3}$ & $X_{2}$, $X_{3}$ & $X_{1}$, $X_{2}$, $X_{3}$ & None\\
\hline
std1 = 0.5\\
Bias & -0.003 & 0.281 & 0.364 & -0.006 & -0.001 & 0.360 & -0.005 & 0.278\\

Variance & 0.031 & 0.021 & 0.047 & 0.021 & 0.041 & 0.037 & 0.029 & 0.037\\

MSE & 0.031 & 0.101 & 0.180 & 0.021 & 0.041 & 0.166 & 0.029 & 0.115\\

std1 = 1.5\\
Bias & 0.008 & 0.855 & 1.009 & 0.007 & 0.004 & 1.013 & 0.008 & 0.847\\

Variance & 0.033 & 0.028 & 0.048 & 0.024 & 0.046 & 0.040 & 0.035 & 0.041\\

MSE & 0.033 & 0.759 & 1.066 & 0.024 & 0.046 & 1.066 & 0.035 & 0.759\\

std2 = 0.5\\
Bias & 0.003 & 0.595 & 0.738 & -0.001 & 0.001 & 0.734 & -0.004 & 0.599\\

Variance & 0.007 & 0.015 & 0.020 & 0.005 & 0.011 & 0.018 & 0.009 & 0.017\\

MSE & 0.007 & 0.369 & 0.564 & 0.005 & 0.011 & 0.557 & 0.009 & 0.376\\

std2 = 1.5\\
Bias & 0.015 & 0.602 & 0.751 & 0.014 & 0.019 & 0.748 & 0.014 & 0.586\\

Variance & 0.106 & 0.055 & 0.146 & 0.090 & 0.147 & 0.122 & 0.121 & 0.137\\

MSE & 0.106 & 0.418 & 0.710 & 0.090 & 0.147 & 0.680 & 0.121 & 0.480\\

std3 = 0.5\\
Bias & -0.003 & 0.698 & 0.732 & -0.004 & -0.002 & 0.732 & -0.003 & 0.691\\

Variance & 0.032 & 0.025 & 0.045 & 0.022 & 0.038 & 0.035 & 0.027 & 0.041\\

MSE & 0.032 & 0.512 & 0.581 & 0.022 & 0.038 & 0.571 & 0.027 & 0.518\\

std3 = 1.5\\
Bias & -0.001 & 0.495 & 0.734 & -0.004 & -0.006 & 0.734 & -0.008 & 5.07\\

Variance & 0.027 & 0.022 & 0.055 & 0.021 & 0.049 & 0.044 & 0.037 & 0.041\\

MSE & 0.028 & 0.267 & 0.593 & 0.021 & 0.049 & 0.582 & 0.038 & 0.518\\

a4 = 0.25\\
Bias & -0.004 & 0.593 & 0.731 & 0.000 & -0.007 & 0.737 & 0.000 & 0.600\\

Variance & 0.032 & 0.025 & 0.049 & 0.023 & 0.044 & 0.039 & 0.033 & 0.040\\

MSE & 0.032 & 0.377 & 0.583 & 0.023 & 0.044 & 0.582 & 0.033 & 0.399\\

a4 = 1\\
Bias & 0.004 & 0.601 & 0.735 & 0.005 & 0.004 & 0.735 & 0.005 & 0.597\\

Variance & 0.033 & 0.027 & 0.052 & 0.023 & 0.047 & 0.042 & 0.035 & 0.040\\

MSE & 0.033 & 0.387 & 0.592 & 0.023 & 0.047 & 0.582 & 0.035 & 0.397\\

b0 = 1\\
Bias & -0.012 & 0.562 & 0.687 & -0.010 & -0.015 & 0.690 & -0.012 & 0.575\\

Variance & 0.034 & 0.021 & 0.052 & 0.022 & 0.047 & 0.039 & 0.032 & 0.041\\

MSE & 0.035 & 0.337 & 0.525 & 0.022 & 0.047 & 0.515 & 0.032 & 0.372\\
\hline
\end{tabular}
\end{table}

\subsection{Experiment 2}

We were not able to replicate the figures of experiment 2.

\FloatBarrier

\section{Discussion}

\subsection{Replicability}

In general, the data-generating mechanism is easy to replicate. It is
clearly described in the original paper and it can be easily coded.
Also, experiment 1 can be replicated with only a few replication degrees
of freedom. However, availability of open source code will improve its
replicability. The initial lack of explanation for the c-statistic or
the issue with the model with splines could have been solved easily if
the code was available. Experiment two is more difficult to replicate,
or, at least, the figures on the paper. The model estimation and the
computation of the measures are similar to experiment one. However, once
these results are obtained, it is necessary to use some transformations
that are not explained at all in the paper. Also, as stated previously,
replicating a complex figure is not an easy task.In fact, we were not
able to do so. The fact that we were not able to replicate experiment 2
does not mean that it is not possible. It is likely that more
experienced researchers in the field would be able to replicate
experiment 2.

\subsection{Replicator degrees of freedom}

In experiment 1, we needed to guess some concepts due to the lack of
information on the article. However, the choices were straightforward
and it we were able to replicate the experiment. In experiment 2, the
lack of information about some concepts and scales did not allow us to
replicate the experiment. Probably, when writing the paper the authors
addressed an audience with a higher experience in the field than the
authors of this report. However, it is always a good practice to write a
paper as self-explanatory as possible. This will help reaching a wider
audience and reduce the replicator degrees of freedom. At the same time,
these technical details are not relevant for understanding the main
concept or the message the authors want to present. They are mainly
necessary for the purpose of replication.

\subsection{Equivalence of results}

The equivalence of results in experiment 1 is extremely good. Despite
using a different R version and code, we were able to obtain very
similar same tables. We cannot judge the results of experiment 2 because
we were not able to obtain the same figures.

\section{Comparison with other pair performing the same replication study}

The other pair was able to perform an even more successful replication.
Both groups were able to replicate experiment 1 with similar issues that
we solved similarly yet independently. On the other hand, the other pair
was able to replicate figure 2 and figure 3 from the original paper and
almost able to replicate figure 4.

A discussion between both groups leaded to the following conclusions:

The order in which the method, measures and results are presented in the
paper can be improved in order to facilitate the understanding and
replicability of the research.

The original paper omit some basic information assuming it is common
knowledge. However, we do not consider it common knowledge. This fact
could explain why one of the groups was able to replicate more than the
other. Actually, in latter versions of the paper they included
explanations of some concepts (e.g.~meaning of c-statistic).

The first experiment is explained with more detail, which makes it
easier to replicate. On the other hand, both groups consider that
experiment two contains some concepts that are not fully explained,
which hurts its replicability.

Both groups consider that the original code of the research could be
available publicly since it is coded in an open source programming
laguage (R), and it does not require new functions that could be under
copyrigth. Publishing the code will improve the replicability and
transparency of the research.

\section{Contributions}

Authors made the following contributions according to the CRediT
framework \url{https://casrai.org/credit/}

Primary Replicator:

\begin{itemize}
\tightlist
\item
  Data Curation\\
\item
  Formal Analysis (supporting)\\
\item
  Investigation\\
\item
  Software\\
\item
  Visualization (lead)\\
\item
  Writing - Original Draft Preparation\\
\item
  Writing - Review \& Editing
\end{itemize}

Co-Pilot:

\begin{itemize}
\tightlist
\item
  Formal Analysis (lead)\\
\item
  Investigation\\
\item
  Software (supporting)\\
\item
  Visualization (supporting)\\
\item
  Validation\\
\item
  Writing - Review \& Editing
\end{itemize}

\newpage

\section*{References}
\begingroup
\hphantom{x}
\setlength{\parindent}{-0.5in}
\setlength{\leftskip}{0.5in}

\hypertarget{refs}{}
\begin{CSLReferences}{1}{0}
\leavevmode\hypertarget{ref-Brookhart2006}{}%
Brookhart, M Alan, Sebastian Schneeweiss, Kenneth J Rothman, Robert J
Glynn, Jerry Avorn, and Til Stürmer. 2006. {``{Variable Selection for
Propensity Score Models}.''} \emph{American Journal of Epidemiology} 163
(12): 1149--56. \url{https://doi.org/10.1093/aje/kwj149}.

\leavevmode\hypertarget{ref-rougier_sustainable_2017-1}{}%
Rougier, Nicolas P, Konrad Hinsen, Frédéric Alexandre, Thomas Arildsen,
Lorena A Barba, Fabien C Y Benureau, C Titus Brown, et al. 2017.
{``{Sustainable Computational Science: The {\{}{\{}ReScience{\}}{\}}
Initiative}.''} \emph{PeerJ Computer Science} 3 (December): e142.
\url{https://doi.org/10.7717/peerj-cs.142}.

\end{CSLReferences}

\FloatBarrier
\endgroup
\newpage

Appendix

\subsection{Code organization}

The code and the files associated are organized in the form of a
research compendium which can be found in the following git repository
\texttt{https://github.com/replisims/Brookhart-2006-FJ}

\begin{verbatim}
## .
## +-- collectResults.R
## +-- computeResults.R
## +-- computeResultsExp2.R
## +-- defs.aux
## +-- defs.tex
## +-- flowchart.PNG
## +-- getResults.R
## +-- Lato-Black.ttf
## +-- Lato-BlackItalic.ttf
## +-- Lato-Bold.ttf
## +-- Lato-BoldItalic.ttf
## +-- Lato-Italic.ttf
## +-- Lato-Regular.ttf
## +-- MainScript.R
## +-- Method_exp1.R
## +-- Method_exp2.R
## +-- Method_new.R
## +-- Method_old.R
## +-- My Collection.bib
## +-- MyDataGeneration.R
## +-- MyEvaluationPC.R
## +-- plots.R
## +-- Preparation.R
## +-- README.md
## +-- references.bib
## +-- replicationFelix.Rproj
## +-- Replisims Juan and Felix.Rmd
## +-- Replisims-Juan-and-Felix.log
## +-- Replisims-Juan-and-Felix.pdf
## +-- Replisims-Juan-and-Felix.Rmd
## +-- Replisims-Juan-and-Felix.tex
## +-- replisims2.Rmd
## +-- ResultsEnvironment.RData
## +-- SimulationAllCells.R
## +-- UbuntuMono-Bold.ttf
## +-- UbuntuMono-BoldItalic.ttf
## +-- UbuntuMono-Italic.ttf
## +-- UbuntuMono-Regular.ttf
## \-- Untitled
##     +-- defs.aux
##     +-- defs.tex
##     +-- flowchart.PNG
##     +-- Lato-Black.ttf
##     +-- Lato-BlackItalic.ttf
##     +-- Lato-Bold.ttf
##     +-- Lato-BoldItalic.ttf
##     +-- Lato-Italic.ttf
##     +-- Lato-Regular.ttf
##     +-- references.bib
##     +-- UbuntuMono-Bold.ttf
##     +-- UbuntuMono-BoldItalic.ttf
##     +-- UbuntuMono-Italic.ttf
##     +-- UbuntuMono-Regular.ttf
##     +-- Untitled.log
##     +-- Untitled.pdf
##     +-- Untitled.Rmd
##     \-- Untitled.tex
\end{verbatim}


Reproducibility Information

This report was last updated on 2022-03-06 16:00:08. The simulation
replication was conducted using the following computational environment
and dependencies:

\FloatBarrier

\begin{verbatim}
## - Session info ---------------------------------------------------------------
##  setting  value                       
##  version  R version 4.0.3 (2020-10-10)
##  os       Windows 10 x64              
##  system   x86_64, mingw32             
##  ui       RTerm                       
##  language (EN)                        
##  collate  Spanish_Spain.1252          
##  ctype    Spanish_Spain.1252          
##  tz       Europe/Berlin               
##  date     2022-03-06                  
## 
## - Packages -------------------------------------------------------------------
##  package        * version    date       lib
##  cachem           1.0.5      2021-05-15 [1]
##  callr            3.7.0      2021-04-20 [1]
##  cli              3.0.0      2021-06-30 [1]
##  crayon           1.4.1      2021-02-08 [1]
##  desc             1.3.0      2021-03-05 [1]
##  devtools         2.4.2      2021-06-07 [1]
##  digest           0.6.27     2020-10-24 [1]
##  dplyr          * 1.0.2      2020-08-18 [1]
##  ellipsis         0.3.2      2021-04-29 [1]
##  evaluate         0.14       2019-05-28 [1]
##  fansi            0.5.0      2021-05-25 [1]
##  fastmap          1.1.0      2021-01-25 [1]
##  fs               1.5.0      2020-07-31 [1]
##  generics         0.1.0      2020-10-31 [1]
##  glue             1.4.2      2020-08-27 [1]
##  htmltools        0.5.1.1    2021-01-22 [1]
##  knitr          * 1.36       2021-09-29 [1]
##  lifecycle        1.0.0      2021-02-15 [1]
##  magrittr         2.0.1      2020-11-17 [1]
##  memoise          2.0.0      2021-01-26 [1]
##  pillar           1.6.1      2021-05-16 [1]
##  pkgbuild         1.2.0      2020-12-15 [1]
##  pkgconfig        2.0.3      2019-09-22 [1]
##  pkgload          1.2.1      2021-04-06 [1]
##  prettyunits      1.1.1      2020-01-24 [1]
##  processx         3.5.2      2021-04-30 [1]
##  ps               1.6.0      2021-02-28 [1]
##  purrr            0.3.4      2020-04-17 [1]
##  R6               2.5.0      2020-10-28 [1]
##  remotes          2.4.0      2021-06-02 [1]
##  RepliSimReport   0.0.0.9000 2022-03-05 [1]
##  rlang            0.4.11     2021-04-30 [1]
##  rmarkdown        2.11       2021-09-14 [1]
##  rprojroot        2.0.2      2020-11-15 [1]
##  sessioninfo      1.1.1      2018-11-05 [1]
##  stringi          1.6.2      2021-05-17 [1]
##  stringr          1.4.0      2019-02-10 [1]
##  testthat         3.0.4      2021-07-01 [1]
##  tibble           3.1.2      2021-05-16 [1]
##  tidyselect       1.1.1      2021-04-30 [1]
##  usethis          2.0.1      2021-02-10 [1]
##  utf8             1.2.1      2021-03-12 [1]
##  vctrs            0.3.8      2021-04-29 [1]
##  withr            2.4.2      2021-04-18 [1]
##  xfun             0.26       2021-09-14 [1]
##  xtable         * 1.8-4      2019-04-21 [1]
##  yaml             2.2.1      2020-02-01 [1]
##  source                                   
##  CRAN (R 4.0.5)                           
##  CRAN (R 4.0.5)                           
##  CRAN (R 4.0.5)                           
##  CRAN (R 4.0.5)                           
##  CRAN (R 4.0.5)                           
##  CRAN (R 4.0.5)                           
##  CRAN (R 4.0.5)                           
##  CRAN (R 4.0.2)                           
##  CRAN (R 4.0.5)                           
##  CRAN (R 4.0.0)                           
##  CRAN (R 4.0.5)                           
##  CRAN (R 4.0.5)                           
##  CRAN (R 4.0.5)                           
##  CRAN (R 4.0.5)                           
##  CRAN (R 4.0.5)                           
##  CRAN (R 4.0.5)                           
##  CRAN (R 4.0.5)                           
##  CRAN (R 4.0.5)                           
##  CRAN (R 4.0.5)                           
##  CRAN (R 4.0.5)                           
##  CRAN (R 4.0.5)                           
##  CRAN (R 4.0.5)                           
##  CRAN (R 4.0.0)                           
##  CRAN (R 4.0.5)                           
##  CRAN (R 4.0.0)                           
##  CRAN (R 4.0.5)                           
##  CRAN (R 4.0.5)                           
##  CRAN (R 4.0.0)                           
##  CRAN (R 4.0.5)                           
##  CRAN (R 4.0.5)                           
##  Github (replisims/RepliSimReport@5f14003)
##  CRAN (R 4.0.5)                           
##  CRAN (R 4.0.5)                           
##  CRAN (R 4.0.5)                           
##  CRAN (R 4.0.0)                           
##  CRAN (R 4.0.3)                           
##  CRAN (R 4.0.0)                           
##  CRAN (R 4.0.5)                           
##  CRAN (R 4.0.5)                           
##  CRAN (R 4.0.5)                           
##  CRAN (R 4.0.5)                           
##  CRAN (R 4.0.5)                           
##  CRAN (R 4.0.5)                           
##  CRAN (R 4.0.5)                           
##  CRAN (R 4.0.5)                           
##  CRAN (R 4.0.2)                           
##  CRAN (R 4.0.0)                           
## 
## [1] C:/Users/jclaramunt/Documents/R/win-library/4.0
## [2] C:/Program Files/R/R-4.0.3/library
\end{verbatim}

The current Git commit details are:

\begin{verbatim}
## Local:    master C:/Users/jclaramunt/surfdrive/replicationFelix
## Remote:   master @ origin (https://github.com/jclaramunt/replicationFandJ)
## Head:     [32b0e4a] 2021-03-18: Last commit
\end{verbatim}


\end{document}
